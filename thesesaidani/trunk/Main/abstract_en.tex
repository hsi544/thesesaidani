This thesis aims to define a design methodology for high performance applications on future embedded processors. These architectures require an efficient usage of their different level of parallelism (fine-grain, coarse-grain), and a good handling of the inter-processor communications and memory accesses.
In order to study this methodology, we have used a target processor which represents this type of emerging architectures, the Cell BE processor. We have also chosen a low level image processing application, the Harris points of interest detector, which is representative of a typical low level image processing application that is highly parallel.
We have studied several parallelisation schemes of this application and we could establish different optimisation techniques by adapting the software to the specific SIMD units of the Cell  processor. We have also developped a library named CELL MPI that allows efficient communication and synchronisation over the processing elements, using a simplified and implicit programming interface. This work allowed us to develop a methodology that simplifies the design of a parallel algorithm on the Cell processor.
We have designed a parallel programming tool named SKELL BE which is based on algorithmic skeletons. This programming model provides an original solution of a meta-programming based code generator. Using SKELL BE, we can obtain very high performances applications that uses the Cell architecture efficiently when compared to other tools that exist on the market.\\
\textbf{Keywords: } Parallel programming, Cell processor, Image processing, Algorithmic skeletons, High performance computing, Meta-programming, embedded processors