Dans cette section, on se propose de comparer l'implementation du m�me algorithme de d�t�ction de points d'inter�ts de Harris sur d'autres architectures parall�les émergentes utilisant eventuellement d'autres mod�les de programmations que le processeur Cell. Les architectures consid�r�es dans cette �tude sont des architectures du type SMP � m�moire partag�e (Intel et PowerPC) ainsi que les cartes graphiques Nvidia et leur langage de programmation CUDA (\emph{Compute Unified Device Architecture}).
\subsection{Les Processeurs Multi-Coeurs}
Ce type de processeurs constitue aujourd'hui le \emph{main stream} en terme de conception d'architectures parall�les et est � la fois le plus diss�min� dans les machines grand public. Le parall�lisme y est pr�sent � plusieurs niveaux, car ces architectures reprennent les concepts des architectures classiques mono-coeurs et dupliquent les coeurs afin d'obtenir un niveau de parall�lisme sup�rieur. La m�moire y est g�n�ralement partag�e et la hi�rarchie m�moire et bas�e sur plusieurs niveaux de caches communs ou pas. Les mod�les et outils de programmations utilis�s pour tirer profit du parall�lisme de ces architectures sont les librairie de \emph{threads} \emph{Pthread} ainsi que les directives de compilation \emph{OpenMP}. En ce qui concerne les optimisations bas-niveau comme celles au niveau des instructions ou certaines optimisations SIMD, elles sont g�n�ralement bien g�r�s par les compilateurs modernes. 
\subsection{Les GPU Nvidia et CUDA}
