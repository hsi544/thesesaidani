\documentclass[twoside,french,a4,12pt,openright]{MYbook}

\usepackage{psfig,epsf,epsfig,eepic,epic}
\usepackage[french]{babel}
\usepackage[latin1]{inputenc}
\usepackage{fancybox}
\usepackage{amsmath,amssymb,amsbsy}
\usepackage{euscript,bm}
\usepackage{theorem}
\usepackage{array,longtable}
\usepackage{times}
\usepackage[gen]{eurosym}
\usepackage{graphicx,graphics}
\usepackage{fancyhdr}
\usepackage{cite}
\usepackage{subfigure}
\usepackage{placeins}
\usepackage{float}
\usepackage[T1]{fontenc}
\usepackage{lmodern}
\usepackage{aeguill}
\usepackage{pdfpages}
\usepackage{verbatim}
\usepackage[Lenny]{fncychap}

%Sonny, Lenny, Glenn, Conny, Rejne and Bjarne
%\usepackage{acronym}
%\usepackage[tight]{minitoc}
%\usepackage{glosstex}
%============ Les fonts ===============================
%\usepackage{bookman}
%\usepackage{helvet}
%\usepackage{lmodern}
%\usepackage{mathptmx}
\usepackage{pandora}
%\usepackage{fourier}
%\usepackage[adobe-utopia]{mathdesign}


%=======================================================
% *******************************
\newtheorem{Lemma}{\em \bf Lemme}[chapter]
% *******************************
\newtheorem{theorem}{\em \bf Th�or�me}[chapter]
% *******************************
\newtheorem{Proposition}{\em \bf Proposition}[chapter]
% *******************************
\newtheorem{Definition}{\em \bf D�finition}[chapter]
% *******************************
\newtheorem{assumption}{\em \bf Hypoth�se}[chapter]
% *******************************

%========== Style du document=========================

\textwidth 16cm

\textheight 23cm

\oddsidemargin 0.5cm

\evensidemargin 0.5cm

\topmargin -1.5cm

\headsep 1cm

\pagestyle{fancy}

\renewcommand{\headrulewidth}{0.4pt}  %Trace un trait de s�paration de largeur 0,4 point. Mettre 0pt pour supprimer le trait.
\renewcommand{\footrulewidth}{0.4pt}

\linespread{1.25}  % l'espacement entre les lignes

%====================================================


%\newcommand{\del}{\stackrel{\rm def}{=}}
\newcommand{\del}{\triangleq}
\newcommand{\av}{{\bm a}}
\newcommand{\bv}{{\bm b}}
\newcommand{\cv}{{\bm c}}
\newcommand{\ev}{{\bm e}}
\newcommand{\fv}{{\bm f}}
\newcommand{\hv}{{\bm h}}
\newcommand{\rv}{{\bm r}}
\newcommand{\sv}{{\bm s}}
\newcommand{\uv}{{\bm u}}
\newcommand{\vv}{{\bm v}}
\newcommand{\wv}{{\bm w}}
\newcommand{\xv}{{\bm x}}
\newcommand{\yv}{{\bm y}}
\newcommand{\zv}{{\bm z}}

\newcommand{\Av}{{\bm A}}
\newcommand{\Bv}{{\bm B}}
\newcommand{\Cv}{{\bm C}}
\newcommand{\Dv}{{\bm D}}
\newcommand{\Ev}{{\bm E}}
\newcommand{\Gv}{{\bm G}}
\newcommand{\Hv}{{\bm H}}
\newcommand{\Iv}{{\bm I}}
\newcommand{\Mv}{{\bm M}}
\newcommand{\Pv}{{\bm P}}
\newcommand{\Qv}{{\bm Q}}
\newcommand{\Rv}{{\bm R}}
\newcommand{\Sv}{{\bm S}}
\newcommand{\Tv}{{\bm T}}
\newcommand{\Uv}{{\bm U}}
\newcommand{\Vv}{{\bm V}}
\newcommand{\Wv}{{\bm W}}
\newcommand{\Xv}{{\bm X}}
\newcommand{\Zv}{{\bm Z}}

\newcommand{\Cc}{\mathcal{C}}
\newcommand{\Dc}{\mathcal{D}}
\newcommand{\Ec}{\mathcal{E}}
\newcommand{\Fc}{\mathcal{F}}
\newcommand{\Gc}{\mathcal{G}}
\newcommand{\Hc}{\mathcal{H}}
\newcommand{\Jc}{\mathcal{J}}
\newcommand{\Kc}{\mathcal{K}}
\newcommand{\Lc}{\mathcal{L}}
\newcommand{\Mc}{\mathcal{M}}
\newcommand{\Nc}{\mathcal{N}}
\newcommand{\Oc}{\mathcal{O}}
\newcommand{\Rc}{\mathcal{R}}
\newcommand{\Qc}{\mathcal{Q}}
\newcommand{\Sc}{\mathcal{S}}
\newcommand{\Wc}{\mathcal{W}}
\newcommand{\Xc}{\mathcal{X}}


\newcommand{\Ccv}{\boldsymbol{\mathcal{C}}}
\newcommand{\Dcv}{\boldsymbol{\mathcal{D}}}
\newcommand{\Ecv}{\boldsymbol{\mathcal{E}}}
\newcommand{\Hcv}{\boldsymbol{\mathcal{H}}}
\newcommand{\Mcv}{\boldsymbol{\mathcal{M}}}
\newcommand{\Pcv}{\boldsymbol{\mathcal{P}}}
\newcommand{\Qcv}{\boldsymbol{\mathcal{Q}}}
\newcommand{\Rcv}{\boldsymbol{\mathcal{R}}}
\newcommand{\Scv}{\boldsymbol{\mathcal{S}}}
\newcommand{\Ucv}{\boldsymbol{\mathcal{U}}}
\newcommand{\Vcv}{\boldsymbol{\mathcal{V}}}
\newcommand{\Xcv}{\boldsymbol{\mathcal{X}}}
\newcommand{\Ycv}{\boldsymbol{\mathcal{Y}}}
\newcommand{\Zcv}{\boldsymbol{\mathcal{Z}}}

\newcommand{\Cf}{\mathfrak{C}}
\newcommand{\Ff}{\mathfrak{F}}
\newcommand{\Gf}{\mathfrak{G}}
\newcommand{\Jf}{\mathfrak{J}}
\newcommand{\Kf}{\mathfrak{K}}

\newcommand{\DELTA}{\boldsymbol{\Delta}}
\newcommand{\Epsilon}{\boldsymbol{\epsilon}}
\newcommand{\Lamb}{\boldsymbol{\Lambda}}
\newcommand{\Gam}{\boldsymbol{\gamma}}
\newcommand{\GAM}{\boldsymbol{\Gamma}}
\newcommand{\Rho}{\boldsymbol{\rho}}
\newcommand{\PI}{\boldsymbol{\Pi}}
\newcommand{\PHI}{\boldsymbol{\Phi}}
\newcommand{\PSI}{\boldsymbol{\Psi}}
\newcommand{\SIGM}{\boldsymbol{\Sigma}}
\newcommand{\XI}{\boldsymbol{\Xi}}

\newcommand{\Es}{\mathbb{E}}

\newcommand{\diag}{{\rm diag}}
\newcommand{\trace}{{\rm trace}}
\newcommand{\off}{{\rm off}}
\newcommand{\rang}{{\rm Range}}

\newcommand{\reffig}[1]     {figure~\ref{#1}}
\newcommand{\reftab}[1]     {tableau~\ref{#1}}


\newcommand{\myline}{\hspace*{-1cm} \rule[1cm]{17cm}{0.5mm}}
%\newcommand{\myline}{\vspace{-3cm} \hrule height 4pt}
\newcommand{\mylineH}{\vspace{-3cm} \hrule height 4pt}
\newcommand{\mylineB}{\vspace{2.5cm} \hrule height 4pt}

\setlength{\parindent}{0cm}

%\newcommand{\clearemptydoublepage}{%
%=================================================================
\begin{document}

%---- Page de titre en fran�ais ----
\input{Titre}

\pagestyle{empty}
%=================================================================
%=========  Page de garde ========================================
%=================================================================
%\begin{titlepage}
%%\vspace{-5.0cm}
%\itshape{
%
%\vspace{0.5cm} \centerline{\textbf{�COLE NATIONALE SUP�RIEURE DES
%T�L�COMMUNICATIONS}} \vspace{0.5cm} \centerline{\textbf{D�PARTEMENT
%TRAITEMENT DU SIGNAL ET DES IMAGES}}
%
%\vspace{0.5cm}
%\begin{figure}[htbp]
%%\hspace{-1.0cm}\includegraphics[width=3.0cm]{univ_paris.eps}\hfill
%\hspace{6.6cm}\includegraphics[width=2.5cm]{./figures/logo}\hfill
%%\hspace{0.2cm}\includegraphics[width=4.0cm]{wavecom.eps}
%\end{figure}}


%\vspace{0.2cm} %\centerline{Rapport Bibliographique sur}
%%\centerline{les m�thodes de s�paration aveugle de sources}
%%\centerline{Dans le cadre de la th�se intitul�e} %\centerline{des Etudes Approfondies}
%%\vspace{0.3cm} \centerline{\large{\textbf{Automatique et Traitement du Signal}}}
%\vspace{1.8cm} {\begin{center} \shadowbox{\LARGE
%\begin{tabular}{ccc} S�paration aveugle de sources audio
%\end{tabular}} \end{center}}
%\vspace{2.0cm}
%\hspace{-0.6cm}Dirig� par: \hspace{5.4cm} Fait par:\\
%\large{Karim ABED-MERAIM} \hspace{2.6cm} \large{Abdeldjalil AISSA-EL-BEY}\\
%\large{Yves GRENIER}
%
%\vspace{4.0cm}\centerline{Juin 2007}
%
%
%\end{titlepage}

%=================================================================
%=========  Fin de la Page de garde ==============================
%=================================================================

%%%%%%%%%%%%%% Page Blanche %%%%%%%%%%%%%%%%%%%%%%%%%%%%%%%%%%%%%%
\newpage
\strut  ~  \mbox{}  \null
\newpage
%%%%%%%%%%%%%%%%%%%%%%%%%%%%%%%%%%%%%%%%%%%%%%%%%%%%%%%%%%%%%%%%%%

\pagestyle{fancy}


\lhead{} \chead{} \lfoot{ENST} \cfoot{\thepage} \rfoot{}
\sloppy
\noindent
%\indent

%\newpage{\pagestyle{empty}\cleardoublepage}}
%=========================================
\setcounter{page}{1}
\frontmatter

\tableofcontents

\listoffigures

\listoftables


%==========================================================
\include{abbreviation}

\include{Notations}



\setcounter{page}{1}
\mainmatter
%==========================================================
%   Chapitre : Introduction g�n�rale
%==========================================================
%\newpage{\pagestyle{empty}\cleardoublepage}}
\include{Intro}
%==========================================================

%==========================================================
%   Chapitre 1 S�paration de sources audio
%==========================================================
%\newpage{\pagestyle{empty}\cleardoublepage}}
\include{Chapt01}
%==========================================================

%==========================================================
%   Chapitre 2 S�paration de sources audio sous-d�termin�
%==========================================================
%\newpage{\pagestyle{empty}\cleardoublepage}}
\include{Chapt02}
%==========================================================


\include{Part01}
%==========================================================
%   Chapitre 3 La SAS Audio utilisant la
%   D�composition Modale: M�langes Instantan�s
%==========================================================
%\newpage{\pagestyle{empty}\cleardoublepage}}
\include{Chapt03}
%==========================================================

%==========================================================
%   Chapitre 4 La SAS Audio utilisant la
%   D�composition Modale: M�langes Convolutifs
%==========================================================
%\newpage{\pagestyle{empty}\cleardoublepage}}
\include{Chapt04}
%==========================================================

\include{Part02}
%==========================================================
%   Chapitre 5 La SAS Audio utilisant la
%   repr�sentation temps-fr�quence :
%   M�langes Instantan�s
%==========================================================
%\newpage{\pagestyle{empty}\cleardoublepage}}
\include{Chapt05}
%==========================================================

%==========================================================
%   Chapitre 6 La SAS Audio utilisant la
%   repr�sentation temps-fr�quence :
%   M�langes Convolutifs
%==========================================================
%\newpage{\pagestyle{empty}\cleardoublepage}}
\include{Chapt06}
%==========================================================

\include{Part03}
%==========================================================
%   Chapitre 7 La SAS Audio utilisant la
%   parcimonie temporelle
%==========================================================
%\newpage{\pagestyle{empty}\cleardoublepage}}
\include{Chapt07}
%==========================================================

%==========================================================
%   Chapitre 8 La SAS Audio utilisant les
%   statistiques d'ordre deux
%==========================================================
%\newpage{\pagestyle{empty}\cleardoublepage}}
\include{Chapt08}
%==========================================================

%==========================================================
%   Chapitre : Conclusion g�n�rale
%==========================================================
%\newpage{\pagestyle{empty}\cleardoublepage}}
\include{Conclusion}
%==========================================================

\appendix

%==========================================================
%   Annexe A : Identifiabilit� au second ordre
%==========================================================
%\newpage{\pagestyle{empty}\cleardoublepage}}
\include{AppendixA}
%==========================================================

%==========================================================
%   Annexe B : D�monstration des th�or�mes
%==========================================================
%\newpage{\pagestyle{empty}\cleardoublepage}}
\include{AppendixB}
%==========================================================
%\includepdf[pages=1-11]{04099541}
%=========================================
%   Bibliographie
%=========================================

\rhead{BIBLIOGRAPHIE}
\addcontentsline{toc}{chapter}{Bibliographie}
%\bibliographystyle{IEEEtran}
%\bibliographystyle{unsrt}
\bibliographystyle{IEEE}

\bibliography{These_biblio}

%=========================================

\end{document}
