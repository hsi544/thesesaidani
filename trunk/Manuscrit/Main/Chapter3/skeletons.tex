La programmation parall�le structur�e qu'on appelle programmation par squelettes algorithmiques \cite{skeletons_cole} restraint l'expression du parall�lisme � la composition d'un nombre pr�d�finit de \emph{patterns} nomm�s squelettes. Les squelettes algorithmiques sont des briques de base g�n�riques, portables et r�utilisables pour lesquelles une impl�mentation parall�le peut exister. Ils sont issues des langages de programmation fonctionnelle. Un syst�me de programmation bas� sur les squelettes fournit un ensemble fixe et relativement limit� de squelettes. Chaque squelette repr�sente une mani�re unique d'exploiter le parall�lisme dans une organisation sp�cifique du calcul, tels que le parall�lisme de donn�es, de t�ches, le \emph{divide-and-conquer} parall�le ou encore le pipeline. En combinant ces \emph{patterns} le d�veloppeur peut construire une sp�cification haut-niveau de son programme parall�le. Le syst�me peut ainsi exploiter cette sp�cification pour la transformation de code, l'exploitant efficace des ressources ou encore le placement.\\
La composition des squelettes peut se faire d'une mani�re non-hi�rarchique en mettant en s�quence les diff�rents blocs en en utilisant des variables temporaires pour sauvegarder les r�sultats interm�diaires, ou alors de mani�re hi�rarchique en imbriquant les fonctions squelette et ce en construisant une fonction compos�e dans laquelle le code de plusieurs squelettes est pass� en param�tre d'un autre squelette. Ceci pr�sente une mani�re �l�gante d'exprimer le parall�lisme multi-niveau.