\documentclass[a4paper,11pt]{article}
\usepackage[latin1]{inputenc}
\usepackage[francais]{babel}
%\usepackage{inputenc}
%\makeindex
\begin{document}
\section{Introduction}
\subsection{Contexte de la th�se}
\begin{itemize}
	\item contexte g�n�ral (Laboratoire, �quipe)
	\item Traitement d'images embarqu� 
	\item n�cessit� de code legacy
	\item Contraintes de traitement temps r�el 
	\item Calcul parall�le 
	\item Projets :  Ocelle, Teraops.
\end{itemize}
\subsection{Architectures Parall�les}
\begin{itemize}
	\item Les niveaux de parall�lisme
	\item les diff�rents types d'architectures: Multi-core, GPU, mod�le m�moire(partag�e, distribu�e), SIMD, SWAR
	\item L'architecture du Cell
\end{itemize}
\subsection{Mod�les de Programmation et Outils de Parall�lisation}
\begin{itemize}
	\item Pr�sentation des diff�rents mod�les de programmation (Shared Memory, Message Passing, Stream processing)
	\item OpenMP (Compilateur  XLC, barcelona CellSS) + MPI (MPI en g�n�ral, MPI microtastk IBM) + Stream Processing (RapidMind, Sequoia)
	\item vectorisation (gcc, xlc) difficult�s sur le Cell
	\item Outils de haut-niveau SPEARS, Gedae
	\item validation outils (quand c'est possible) avec code simple (addition matricielle).
\end{itemize}
\section{Optimisation des transferts et du contr�le}
\begin{itemize}
	\item Tiling 1D, 2D
	\item bord, cha�nage d'op�rateur, fusion.
	\item double buffering.
	\item Outils: Cell\_MPI, d�p�t APP( peut �tre dans le contexte aussi) 
\end{itemize}
\section{Parall�lisation de Code de Traitement d'Images}
\subsection{Algorithmes R�gulier}
\subsubsection{Harris points d'int�r�t}
\begin{itemize}
	\item Pr�sentation de l'algorithme
	\item Algorithme repr�sentatif du traitement d'images bas niveau.
	\item difficult�s li�es � l'impl�mentation, adaptation au contraintes du Cell, jeux d'instruction( floats) , DMA.... 
	\item Mod�le de parall�lisation
	\item Mesure de performances
	\item Mod�le analytique pour la pr�diction de performances
\end{itemize}
\subsubsection{Un autre algo r�gulier (Sigma Delta, ..)}
\begin{itemize}
	\item m�morisation dans SPE ou pas.
\end{itemize}
\subsubsection{Un autre algo irr�gulier ECC}
\begin{itemize}
	\item load balancing
\end{itemize}
\section{Conclusion}


\end{document}